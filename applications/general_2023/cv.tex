\documentclass[mm, 10pt]{res} % Use the res.cls style, the font size can be changed to 11pt or 12pt here
\usepackage[T1]{fontenc}   
\usepackage[left=15mm, right=20mm, top=10mm ,bottom=10mm]{geometry}
%\usepackage{helvet} % Default font is the helvetica postscript font
\usepackage{array}
\usepackage{hyperref}


%\usepackage{newcent} % To change the default font to the new century schoolbook postscript font uncomment this line and comment the one above
\usepackage[rm,sfdefault]{roboto}
\usepackage{xcolor}

\setlength{\textwidth}{5.85in} % Text width of the document
\definecolor{color1}{rgb}{0.4,0.5,0.3}% green

\usepackage{enumitem}
\setitemize{noitemsep,topsep=0pt,parsep=0pt,partopsep=0pt}

\newcommand{\ecvitem}[2]{{\robotocondensed #2}

}

\newcommand{\skilla}[1]{
\begin{tabular}{>{\raggedleft}m{1.6cm} l}
#1&\hspace{-0.5em}{\Large{\color{color1}$\bullet$}{\color{gray!50}$\bullet\bullet$}} \\
\end{tabular}
}
\newcommand{\skillb}[1]{
\begin{tabular}{>{\raggedleft}m{1.6cm} l}
#1&\hspace{-0.5em}{\textbf{\Large{\color{color1}$\bullet\bullet$}{\color{gray!50}$\bullet$}}} \\
\end{tabular}
}
\newcommand{\skillc}[1]{
\begin{tabular}{>{\raggedleft}m{1.6cm} l}
#1&\hspace{-0.5em}{\textbf{\Large{\color{color1}$\bullet\bullet$}{\color{color1}$\bullet$}}} \\
\end{tabular}
}

\newcommand{\expbox}[1]{\vspace{-1.8em}
\noindent\begin{tcolorbox}[boxsep=0pt,top=3pt,left=8pt,right=0pt, bottom=2pt,arc=0pt,auto outer arc,colback=color1!15,colframe=color1!10]
\begin{minipage}[r]{0.95\linewidth} 
\small
#1 
\normalsize
\end{minipage}
\end{tcolorbox}
}

\newcommand{\tinyexpbox}[1]{\vspace{-1.8em}
\noindent\begin{tcolorbox}[boxsep=0pt,top=3pt,left=8pt,right=0pt, bottom=2pt,arc=0pt,auto outer arc,colback=color1!15,colframe=color1!10]
\begin{minipage}[r]{0.95\linewidth} 
\footnotesize
#1 
\normalsize
\end{minipage}
\end{tcolorbox}
}


\newcommand{\cvitem}[2]{{\sl #1} \hfill \textbf{#2}\\}
\newcommand{\detcvitem}[3]{{\sl #1} \hfill \textbf{#2}\\{\hspace*{1em} \small \robotocondensed #3}\\}
\newcommand{\project}[5]{{\sl #1} \hfill \textbf{#2}\\ 
\textbf{Partners:} #3\\
\textbf{Decription:} \small #4\\
\textbf{My role:}\vspace{-1em} #5\par}

\usepackage{fontawesome}
\usepackage[most]{tcolorbox}


\begin{document}

 \hspace{-8.5em} { \Large \robotoslab Giulio Luzzati, Ph.D.}% Your name at the top
 \vspace{-0.5em}

\moveleft\hoffset\vbox{\color{color1}\hrule width\resumewidth height 1pt}\smallskip % Horizontal line after name; adjust line thickness by changing the '1pt'
\vspace{-2em}
%----------------------------------------------------------------------------------------

\begin{resume}
\robotoslab

\section{Personal Information}
Principal DSP Engineer\\
{\robotocondensed 1b Mill Road, Haddenham, Ely} \\
Phone: {\robotocondensed +44 07397 791131, +39 3756888072}\\
e-mail: {\robotocondensed giulio.luzzati@live.it}\\
LinkedIn: {\url{https://it.linkedin.com/in/giulio-luzzati-9bb79245}}\\
Born: Genova, Italy
\vspace{-1em}
\section{In a\\Nutshell}~
\expbox{I am a software engineer with a strong scientific background. My drive is to design and analyse systems, understand what makes them tick and how they could be better. I strive to frame my actions holistically, and I have a track record of delivering solid, dependable, and (re)usable solutions.}
\vspace{-1em}
\section{Education}
\textbf{University of Genova,} Genova, Italy\\
\cvitem{Ph.D. in Computer Science}{Apr 2016}\vspace{-1em}\\
\small Thesis topics: resource allocation, communication networks, signal processing \normalsize\\
\cvitem{Professional Engineering Qualification}{Oct 2012}\vspace{-1em}\\
\cvitem{M.Sc. in Telecommunication Engineering}{May 2012}\vspace{-1em}\\	
\vspace{-1em}

\section{Technical \\Skills}
\begin{tabular}{ccc}
\skillc{C}&
\skilla{C++}&
\skillc{Python}\\
\skillb{Matlab}&
\skillb{Latex}&
\skillb{Shell}\\
\skillb{CI/CD}&
\skillc{git}&
\skillb{Docker}\\
\end{tabular}

\section{Scientific \\Skills}
\footnotesize
\begin{tabular}{l}
Signal processing\\
Computer networks\\
Statistics, data science \\
Mathematical optimization \\
Working knowledge in machine learning \\
\end{tabular}

\section{Professional \\and Academic Experience} 

\textbf{Cambridge Touch Technologies (CTT)}, Cambridge, United Kingdom\\
\cvitem{DSP Team Leader}{Jan 2022 - currently}
\cvitem{Principal DSP Engineer}{Jul 2021 - Dec 2021}
\cvitem{Senior DSP Engineer}{Oct 2018 - Jun 2021}
\expbox{CTT is a technology scale-up, whose core business is providing a cost effective technology to add touch-pressure sensing capabilities to display and non-display surfaces. 
At CTT I currently lead the research, development and implementation of the algorithms that are part of the core of the company IP.

The tech stack I contributed to design and implement comprises Matlab and Python for reference implementations, demos, scripts and glue code, and C for low level, embedded applications.

Some major contributions over the years
\begin{itemize}
 \item developed the CI ecosystem, relying on the Gitlab platform and state of the art dockerised build and test environments on a local cloud.
 \item implemented real time visualisations and developing environments to demonstrate, familiarise with, and tune DSP algorithms
 \item provided feedback to help develop the sensor technology
 \end{itemize}
}

\vspace {-0.3em}
\textbf{5G Innovation Centre}, Guildford, United Kingdom\\
\cvitem{Senior Software Engineer}{Nov 2017 - Oct 2018}
\expbox{The 5GIC is a research centre within the University of Surrey, featuring one of the biggest 5G testbeds in the UK with research and standardization activity in close partnership with some of the largest players in the field. The 5GIC's testbed, a complete cellular network, showcases several research ideas, from the physical layer to network concepts.
At the 5GIC I was part of the core network team. My contribution as a software engineer was to develop and maintain the codebase for the core network.}

\vspace {-0.3em}
\textbf{AKYA ltd}, Swindon, United Kingdom\\
\cvitem{DSP Software Engineer}{Dec 2016 - Nov 2017}
\expbox{AKYA's core business was a novel ``dynamically reconfigurable logic'' approach to hardware design, for low power/low cost applications, unlike e.g. FPGA, providing ``just enough'' reconfigurability to meet design requirements. 
At AKYA I contributed as a software engineer to develop a framework that synthesises RT logic from a high-level description of the hardware. My role was to design and develop algorithms and software components that expand and integrate the existing framework, as well as creating tools for testing and data visualization.}

\vspace {-0.3em}
\textbf{DSP Lab, University of Genova}, Genova, Italy\\
\cvitem{Post Doctoral Research Fellow}{Jan 2016 - Nov 2016}
\cvitem{Ph.D. Student}{Jan 2013 - Dec 2015}
\cvitem{Research Fellow}{Oct 2012 - Dec 2012}
\expbox{During my academic experience at the DSP Labs, as Ph.D. student and then research fellow, I carried out academic (research and teaching) activity, along with projects in collaboration with SMEs and some of the key industries of Italy's communications and tech (Telecom Italia, Leonardo). My main area of research were resource allocation and mathematical optimization in communication networks and in signal processing}
% \textbf{National Council of Research (CNR)}, Genova, Italy\\
%{\sl Internship, ISSIA group,  Bachelor Thesis research. } \hfill{\bf  2008}


\footnotesize
\begin{tabular}{lll}
Software Architecture & Build Systems & CI, DevOps \\
Containerization & Embedded Programming & Basic Hardware Diagnostics \\
Automated Testing & Agile Software Development & GNU/Linux OS \\
\end{tabular}
\normalsize
%\section{Relevant \\Projects}
%Touch Model \\
\cvitem{Individual Project, Cambridge Touch Technologies}{2020}
\tinyexpbox{A research on human touch interaction models, this project provides experimental data and insights to support and direct hardware and algorithm design decisions.
I designed and prototyped the hardware and software needed to provide a generalized ``experiment'' platform, to collect touch force profiles for different use cases.
The second part was focused on processing and analysing the gathered data, to refine insights from the aggregate.
The outcome from this phase of the project is parametrized mathematical models for several basic touch interaction modes.
The project's results have been used to drive the design of silicon.
}

\vspace{-1em}
%Signal Injector for Pressure Sensors\\
\cvitem{Project manager, Cambridge Touch Technologies}{2019}
\tinyexpbox{The goal of the project was to realize a signal injector, able to ``simulate'' the behaviour of the physical sensor. An array of DACs at its core, the deliverable consisted in a small box, controlled via REST api, able to physically interface to the pressure sensor amplifier. My role in the project:
\begin{itemize}
\item specified the requirements 
\item designed the high level architecture  
\item implemented driver, API 
\item integration (as slave component) with other existing diagnostic tools
\end{itemize}

The tool proved to be useful and usable, and after the prototype, several units of the tool have been commissioned and are in use by the hardware team.
}

\vspace{-1em}
%Quick Http Messages \\
\cvitem{Individual Project, 5GIC}{2018}
\tinyexpbox{This project was a proof of concept to provide a barebone framework to test ideas and quickly prototype in service based architectures (SbA), such as the 5G core network. The core idea was to implement a simple HTTP server over UDP, to minimize latency. Source code for this project available at \url{https://github.com/giuliol/quick-http-messages}.
Additionally, I realized a simple automatic generator able to parse 3GPP OpenAPI compliant YML for the 5G SbA and generate stubs for their REST APIs. 
}

\vspace{-1em}

%\hfill
%\small
\pagebreak
%\section{Research \& Publications}

\ecvitem{2016}{Igor Bisio, Fabio Lavagetto, Giulio Luzzati and Andrea Sciarrone, ``A Novel Active Warden Technique for Image Steganography'', accepted IEEE GLOBECOM 2016. }
\ecvitem{2016}{I. Bisio, A. Fedeli, F. Lavagetto, G. Luzzati, M. Pastorino, A. Randazzo, and E. Tavanti, ``Brain Stroke Detection by Microwave Imaging Systems: Preliminary Two-Dimensional Numerical Simulations'', submitted to 2016 IEEE International Conference on Imaging Systems and Techniques (IST 2016)}
\ecvitem{2016}{Igor Bisio, Alessandro Fedeli, Fabio Lavagetto, Giulio Luzzati, 
Matteo Pastorino,  Andrea Randazzo, and Emanuele Tavanti, ``Hemorrhagic Brain Stroke Detection by using Microwaves: Preliminary Two-dimensional Reconstructions'', IV Convegno Nazionale ``Interazione tra Campi Elettromagnetici e Biosistemi'', Milano, 4-6 July 2016. }

\ecvitem{2016}{ Igor Bisio, Fabio Lavagetto, Giulio Luzzati, ``Cooperative Application Layer Joint Video Coding in the Internet of Remote Things'', submitted to the IEEE Internet of Things Journal.}

\ecvitem{2015}{Igor Bisio, Giulio Luzzati and Andrea Sciarrone, ``Cell-ID Meter App: a Tester for Coverage Maps Localization Proofs in Forensic'' Investigations, 7th IEEE International Workshop on Information Forensics and SecurityRome, Italy, 16-19 November, 2015}
\ecvitem{2015}{Igor Bisio and Stefano Delucchi and Fabio Lavagetto and Giulio Luzzati and
Mario Marchese, ``Cooperative Application Layer Joint Coding and Rate Allocation Techniques for Video Transmissions over Satellite Channels through Smartphones'', accepted to IEEE ICC 2015 SAC - Satellite and Space Communications (ICC'15 (01) SAC6-SSC)}
\ecvitem{2014}{Igor Bisio, Fabio Lavagetto, Giulio Luzzati, Mario Marchese, ``Smartphones Apps Implementing a Heuristic Joint Coding for Video Transmissions over Mobile Networks'', International Journal of Mobile Networks and Applications (MONET).}
\ecvitem{2014}{Igor Bisio, Aldo Grattarola, Fabio Lavagetto, Giulio Luzzati, Mario Marchese, ``Application Layer Source-Channel Video Coding for Transmission with Smartphones over Satellite Channel'', Proc. The Sixth International Conference on Advances in Satellite and Space Communications (SPACOMM), February 23 - 27, 2014 - Nice, France.}
\ecvitem{2014}{Igor Bisio, Fabio Lavagetto, Giulio Luzzati, Mario Marchese,  ``Smartphones Apps Implementing a Heuristic Joint Coding for Video Transmissions over Mobile Networks'', 6th International Conference on Personal Satellite Services, July 2014, Genoa, Italy}
\ecvitem{2014}{Igor Bisio, Aldo Grattarola, Fabio Lavagetto, Giulio Luzzati, Mario Marchese, ``Performance Evaluation of Application Layer Joint Coding for Video Transmission with Smartphones Over Terrestrial/Satellite Emergency Networks'', Proc. IEEE International Conference on Communications 2014, ICC 2014, 10 – 14 June 2014, Sydney, Australia - Best Paper Award winner}
\ecvitem{2012}{Bisio, I.; Delfino, A.; Luzzati, G.; Lavagetto, F.; Marchese, M.; Fra, C.; Valla, M., ``Opportunistic estimation of television audience through smartphones,'' Performance Evaluation of Computer and Telecommunication Systems (SPECTS), 2012 International Symposium on , vol., no., pp.1,5, 8-11 July 2012}


\end{resume}
\end{document}
