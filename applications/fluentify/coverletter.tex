\documentclass[mm, 10pt]{res} % Use the res.cls style, the font size can be changed to 11pt or 12pt here
\usepackage[T1]{fontenc}   
\usepackage[left=5mm, right=20mm, top=10mm ,bottom=10mm]{geometry}
%\usepackage{helvet} % Default font is the helvetica postscript font
\usepackage{array}
\usepackage{hyperref}


%\usepackage{newcent} % To change the default font to the new century schoolbook postscript font uncomment this line and comment the one above
\usepackage[rm,sfdefault]{roboto}
\usepackage{xcolor}

\setlength{\textwidth}{5.85in} % Text width of the document
\definecolor{color1}{rgb}{0.4,0.5,0.3}% green

\usepackage{enumitem}
\setitemize{noitemsep,topsep=0pt,parsep=0pt,partopsep=0pt}

\newcommand{\ecvitem}[2]{{\robotocondensed #2}

}

\newcommand{\skilla}[1]{
\begin{tabular}{>{\raggedleft}m{1.6cm} l}
#1&\hspace{-0.5em}{\Large{\color{color1}$\bullet$}{\color{gray!50}$\bullet\bullet$}} \\
\end{tabular}
}
\newcommand{\skillb}[1]{
\begin{tabular}{>{\raggedleft}m{1.6cm} l}
#1&\hspace{-0.5em}{\textbf{\Large{\color{color1}$\bullet\bullet$}{\color{gray!50}$\bullet$}}} \\
\end{tabular}
}
\newcommand{\skillc}[1]{
\begin{tabular}{>{\raggedleft}m{1.6cm} l}
#1&\hspace{-0.5em}{\textbf{\Large{\color{color1}$\bullet\bullet$}{\color{color1}$\bullet$}}} \\
\end{tabular}
}

\newcommand{\expbox}[1]{\vspace{-1.8em}
\noindent\begin{tcolorbox}[boxsep=0pt,top=3pt,left=8pt,right=0pt, bottom=2pt,arc=0pt,auto outer arc,colback=color1!15,colframe=color1!10]
\begin{minipage}[r]{0.95\linewidth} 
\small
#1 
\normalsize
\end{minipage}
\end{tcolorbox}
}

\newcommand{\tinyexpbox}[1]{\vspace{-1.8em}
\noindent\begin{tcolorbox}[boxsep=0pt,top=3pt,left=8pt,right=0pt, bottom=2pt,arc=0pt,auto outer arc,colback=color1!15,colframe=color1!10]
\begin{minipage}[r]{0.95\linewidth} 
\footnotesize
#1 
\normalsize
\end{minipage}
\end{tcolorbox}
}


\newcommand{\cvitem}[2]{{\sl #1} \hfill \textbf{#2}\\}
\newcommand{\detcvitem}[3]{{\sl #1} \hfill \textbf{#2}\\{\hspace*{1em} \small \robotocondensed #3}\\}
\newcommand{\project}[5]{{\sl #1} \hfill \textbf{#2}\\ 
\textbf{Partners:} #3\\
\textbf{Decription:} \small #4\\
\textbf{My role:}\vspace{-1em} #5\par}


\newlength{\seplinewidth}
\newlength{\seplinesep}
\setlength{\seplinewidth}{1mm}
\setlength{\seplinesep}{2mm}
\newcommand*{\sepline}{%
  \par
  \vspace{\dimexpr\seplinesep+.5\parskip}%
  \cleaders\vbox{%
    \begingroup % because of color
      \color{color1}%
      \hrule width\linewidth height\seplinewidth
    \endgroup
  }\vskip\seplinewidth
  \vspace{\dimexpr\seplinesep-.5\parskip}%
}


\usepackage[most]{tcolorbox}

\usepackage{fontawesome}
\usepackage[document]{ragged2e}




\begin{document}


\begin{resume}
\robotoslab
{ \color{color1} \Large \robotoslab Giulio Luzzati}% Your name at the top
\vspace{2em}
\sepline

\raggedleft
{\robotocondensed 
{\color{color1}\faMapMarker}~:~1b Mill Road, Haddenham, Ely, United Kingdom} \\
{\color{color1}\faPhone}~:~{\robotocondensed +44 07397 791131}\\
{\color{color1}\faEnvelope}~:~{\robotocondensed giulio.luzzati@live.it}\\
{\color{color1}\faLinkedinSquare}~:~{\url{https://it.linkedin.com/in/giulio-luzzati-9bb79245}}\\

\justify

\vspace{2em}

\sepline

\vspace{5em}

Hello Fluentify,
\vspace{4em}

I'm Giulio, a software engineer with a strong scientific background.

I started my carreer in the field of academic research, then ventured out to the UK in search for new challenges. 
In these past years, I've got the chance to work in few different places: after my Ph.D, I've started off as a DSP Engineer, and gradually grown into a lead role.

I'm currently based in the UK, but considering relocating to Italy and looking for the right occasion that would align with my plans.

This is where I learned about this opportunity with Fluentify - What I found particularly interesting about the job post is being able to work in an international team, 
 the prospect of working with big data pipelines, and the chance to be exposed to NLP in future projects. I believe, among other things, my experience in building microservices and automations could make me particularly
effective in taking decisions when dealing with entities that operate in complex ecosystems and pipelines.

%My background is signal processing and algorithms - that means what I've been striving for in the past years, is finding the best ways to analyse signals, condition them, and understand the fundamental semantics of the nugget of information buried under the noise. 
%As an engineer, I think there is great value in being very opinionated, but ultimately equally highly pragmatic - know the rules and know when to break them.


\vspace{2em}

I'm looking forward to get the chance to better know each other, 

\vspace{4em}

Giulio




\end{resume}
\end{document}
