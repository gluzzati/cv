\documentclass{if-beamer}

% --------------------------------------------------- %
%                  Presentation info	              %
% --------------------------------------------------- %
\title[]{}
\author{Giulio Luzzati}
\date{}
\subject{Presentation} % metadata

% --------------------------------------------------- %
%                    Title + Schedule                 %
% --------------------------------------------------- %
\newcommand{\cvitem}[2]{{\sl #1} \hfill \textbf{#2}\\}
\begin{document}

%please prepare a few slides to introduce yourself focusing on your
%background, skills and relevant past/ongoing projects.

%1- education 
% - fingerprint
% - tesi

%1b postdoc
% - caschetto

%2- akya 

%3- 5GIC
%  - qhm

%4- CTT
% - IAN
% - STK
% - HPM

\section{Profile}

\begin{frame}
\small
My education


\begin{itemize}
\item 2012 M.Sc. in Telecommunication Engineering (thesis on audio fingerprinting)
\item 2016 Ph.D. in Computational Intelligence
\end{itemize}

My tools and skills


\begin{itemize}
\item Software
\begin{itemize}
\item Software architecture and development
\item Build systems
\item Unit testing and continuous integration
\item Embedded (mostly bare metal)
\item Favorite languages
\begin{itemize}
\item C (simple rules, easy to establish coding standards, agnostic)
\item Python (fast prototyping, huge community)
\end{itemize}
\end{itemize}
\item Linux OS, Yocto
\item Computer networks and protocols
\item Electronics, basics (i.e., not afraid to grab a scope to debug inter-ic communications)
\end{itemize}
My scientific background
\begin{itemize}
\item Statistics
\item Information theory
\item Signal processing (audio, signal)
\item Mathematical optimization
\end{itemize}
\end{frame}

\section{Research Activity}

\begin{frame}
\begin{itemize}
\item Application layer joint source/channel coding
\begin{itemize}
\item Realized Android apps performing adaptive source/channel encoded video streams
\item Co-authored scientific publications
\end{itemize}
\item TV channel detection via audio fingerpritning
\begin{itemize}
\item C implementation of live TV audio stream fingerprint computation and matching
\item Ported fingerprint algorithm to iOS
\item Research and optimization of system parameters
\item Co-authored scientific publications
\end{itemize}
\item Brain stroke detection via microwave imaging
\begin{itemize}
\item State of the art survey
\item Implementation of glue code for data manipulation
\item Preliminary investigation on neural network detection (Google Tensorflow)
\item Co-authored scientific publications
\end{itemize}
\end{itemize}
\end{frame}

\iffalse
\subsection{Application Layer Joint Coding}
\begin{frame}
Application Layer Joint Coding


\begin{itemize}
\item Starting Point: application layer channel coding (ALC) adds redundancy and works alongside lower levels FEC
\item $[1]$ There is a region (lossy networks) in which ALC can be detrimental
\item How would source coding (at the application layer) affect the boundary region?
\end{itemize}
\end{frame}


\subsection{TV Audience Estimation via Audio Fingerprinting}
\begin{frame}
TV Audience Estimation via Audio Fingerprinting


\begin{itemize}
\item Starting Point: audio fingerprinting techniques are very good at detecting a specific ``realization'' of an audio signal 
\item Easy to perform in static databases (e.g., songs). Live audio feeds are trickier
\item Live ``correlation'' of audio fingerprints 
\end{itemize}
\end{frame}


\subsection{Brain Stroke Detection via Microwave Imaging}
\begin{frame}
Brain Stroke Detection via Microwave Imaging


\begin{itemize}
\item Starting Point: hemhorragic/occlusion based strokes need opposite treatment. Microwave analysis is able to reveal details about the dielectric properties of the inside of materials.
\item Analytical inversion of the scattering problem is computationally intractable 
\item Use a Gauss-Newton approach and assumptions to find plausible dielectric profile that can originate the measured scattering matrix
\item Directly feed the scattering matrix parameters to simple neural networks
\end{itemize}
\end{frame}
\fi

\section{Industrial Experience}
\subsection{Akya}
\iffalse
\begin{frame}
AKYA's core business was a novel ``dynamically reconfigurable logic'' approach to hardware design, for low power/low cost applications, unlike e.g. FPGA, providing ``just enough'' reconfigurability to meet design requirements. 
At AKYA I contributed as a software engineer to the codebase of a framework to synthesise logic from a high-level description of the hardware. My role was to design and develop algorithms and software components that expand and integrate the existing framework, as well as creating tools for testing and data visualization
\end{frame}
\fi
\subsection{QHM}
\begin{frame}
Quick Http Messages, developed at 5GIC,.


\begin{itemize}
\item HTTP over UDP 
\item playground for SoA (RESTful API services)
\item core messaging implemented
\item initial version of automatic generator (from OpenAPI yml files) 
\item available at \url{https://github.com/giuliol/quick-http-messages}
\end{itemize}
\end{frame}

\subsection{IAN}
\begin{frame}

Touchscreen analyser. Developed at CTT, 2019


\begin{itemize}
\item diagnosing manifacturing defects of pressure sensing enabled touch panels
\item OTS VNA, custom serial operated switch board
\item Impedance spectroscopy to find anomalies
\item My contribution
\begin{itemize}
\item Designed software and data architecture
\item Implemented most of the software
\item Designed data analysis algorithm and metrics to highlight anomalies
\item Automatic report generation
\end{itemize}
\item prototype output validated against known dataset, flagged previously unknown defects.
\item second iteration in design
\end{itemize}
\end{frame}

\subsection{STK}
\begin{frame}
Signal Injector for Pressure Sensors. Developed at CTT, 2019

\begin{itemize}
\item inject signal into amplifier boards, to isolate analog frontend and downstream
\item maximally usable - a self contained box with easy to use web-app to control it
\item My contribution
\begin{itemize}
\item Contributed to requirements specifications
\item Designed functional architecture
\item Designed and implemented drivers, software and communication protocols
\item Integration (as slave component) with other existing diagnostic tools
\end{itemize}
\item Very positive feedback, currently in use by the hardware team
\item Second iteration under development
\end{itemize}
\end{frame}


\subsection{HPM}
\begin{frame}

Human touch interaction model, Developed at CTT, ongoing

\begin{itemize}
\item Provide mathematical models for touch interactions (e.g., force profile as a function of time)
\item Identify assumptions and applicability
\item Gather experimental data
\item Fit mathematical curves
\item My contribution
\begin{itemize}
\item Designed and realized acquisition tool (arduino + strain gauge + haptics)
\item Implemented firmware and software to ``gamify'' the experiment
\item Designed mathematical curves to fit the ``average'' force profiles
\end{itemize}
\item Test machines are able to simulate realistic touches using parametrised mathematical models
\item Design choices can now be strongly backed up by the study
\item Experiment platform is generic and extensible, second study (effects of processing delay) has been commissioned and ongoing
\end{itemize}
\end{frame}


\end{document}

