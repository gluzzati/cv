\documentclass[mm, 10pt]{res} % Use the res.cls style, the font size can be changed to 11pt or 12pt here
\usepackage[T1]{fontenc}   
\usepackage[left=15mm, right=20mm, top=10mm ,bottom=10mm]{geometry}
%\usepackage{helvet} % Default font is the helvetica postscript font
\usepackage{array}
\usepackage{hyperref}


%\usepackage{newcent} % To change the default font to the new century schoolbook postscript font uncomment this line and comment the one above
\usepackage[rm,sfdefault]{roboto}
\usepackage{xcolor}

\setlength{\textwidth}{5.85in} % Text width of the document
\definecolor{color1}{rgb}{0.4,0.5,0.3}% green

\usepackage{enumitem}
\setitemize{noitemsep,topsep=0pt,parsep=0pt,partopsep=0pt}

\newcommand{\ecvitem}[2]{{\robotocondensed #2}

}

\newcommand{\skilla}[1]{
\begin{tabular}{>{\raggedleft}m{1.6cm} l}
#1&\hspace{-0.5em}{\Large{\color{color1}$\bullet$}{\color{gray!50}$\bullet\bullet$}} \\
\end{tabular}
}
\newcommand{\skillb}[1]{
\begin{tabular}{>{\raggedleft}m{1.6cm} l}
#1&\hspace{-0.5em}{\textbf{\Large{\color{color1}$\bullet\bullet$}{\color{gray!50}$\bullet$}}} \\
\end{tabular}
}
\newcommand{\skillc}[1]{
\begin{tabular}{>{\raggedleft}m{1.6cm} l}
#1&\hspace{-0.5em}{\textbf{\Large{\color{color1}$\bullet\bullet$}{\color{color1}$\bullet$}}} \\
\end{tabular}
}

\newcommand{\expbox}[1]{\vspace{-1.8em}
\noindent\begin{tcolorbox}[boxsep=0pt,top=3pt,left=8pt,right=0pt, bottom=2pt,arc=0pt,auto outer arc,colback=color1!15,colframe=color1!10]
\begin{minipage}[r]{0.95\linewidth} 
\small
#1 
\normalsize
\end{minipage}
\end{tcolorbox}
}

\newcommand{\tinyexpbox}[1]{\vspace{-1.8em}
\noindent\begin{tcolorbox}[boxsep=0pt,top=3pt,left=8pt,right=0pt, bottom=2pt,arc=0pt,auto outer arc,colback=color1!15,colframe=color1!10]
\begin{minipage}[r]{0.95\linewidth} 
\footnotesize
#1 
\normalsize
\end{minipage}
\end{tcolorbox}
}


\newcommand{\cvitem}[2]{{\sl #1} \hfill \textbf{#2}\\}
\newcommand{\detcvitem}[3]{{\sl #1} \hfill \textbf{#2}\\{\hspace*{1em} \small \robotocondensed #3}\\}
\newcommand{\project}[5]{{\sl #1} \hfill \textbf{#2}\\ 
\textbf{Partners:} #3\\
\textbf{Decription:} \small #4\\
\textbf{My role:}\vspace{-1em} #5\par}


\usepackage[most]{tcolorbox}


\newcommand\keyword[2][]{\tikz[overlay]\node[fill=color1!10,inner sep=1.8pt, anchor=text, rectangle, rounded corners=1.2mm, #1] {#2};\phantom{#2}}

\begin{document}

 \hspace{-8.5em} { \Large \robotoslab Giulio Luzzati, Ph.D.}% Your name at the top
 \vspace{-0.5em}

\moveleft\hoffset\vbox{\color{color1}\hrule width\resumewidth height 1pt}\smallskip % Horizontal line after name; adjust line thickness by changing the '1pt'
\vspace{-2em}
%----------------------------------------------------------------------------------------

\begin{resume}
\robotoslab

\section{Personal Information}
{\robotocondensed 1b Mill Road, Haddenham, Ely} \\
Phone: {\robotocondensed +44 07397 791131, +39 3756888072}\\
E-mail: {\robotocondensed giulio.luzzati@live.it}\\
LinkedIn: {\url{https://it.linkedin.com/in/giulio-luzzati-9bb79245}}\\
Citizenship: Italian
\vspace{-1em}
\section{In a\\Nutshell}~
\expbox{I am a software engineer with a strong scientific background. My drive is to design and analyse systems, understand what makes them tick and how they could be better. 
I put great effort in thinking pragmatically and holistically, and I have a track record of delivering solid and (re)usable solutions.}
\vspace{0.5em}
\moveleft\hoffset\vbox{\color{color1}\hrule width\resumewidth height 1pt}\smallskip % Horizontal line after name; adjust line thickness by changing the '1pt'
\vspace{0.5em}

\section{Languages \\and Tools}
\begin{tabular}{lll}
\skillc{C}&
\skilla{C++}&
\skillc{Python}\\
\skillb{Matlab}&
\skillb{Latex}&
\skillb{Shell}\\
\skillb{CI/CD}&
\skillc{git}&
\skillb{Docker}\\
\end{tabular}

\section{Scientific \\Skills}
\footnotesize
\begin{tabular}{cc}
\keyword{Signal processing}&
\keyword{Computer networks} \\ 
\keyword{Statistics, data science} &
\keyword{Mathematical optimization} \\
\keyword{Working knowledge in machine learning}
\end{tabular}
\normalsize
\vspace{0.25em}
\section{Technical \\Skills}
\footnotesize
\begin{tabular}{ccc}
\keyword{Software Architecture} & \keyword{Build Systems} & \keyword{CI, DevOps} \\
\keyword{Embedded Programming} & \keyword{Basic Embedded Systems Debugging} & \keyword{GNU/Linux OS} \\
\keyword{Microservices and REST APIs} & ~ & ~ \\
\end{tabular}
\normalsize


\section{Education}
\textbf{University of Genova,} Genova, Italy\\
\cvitem{Ph.D. in Computer Science}{Apr 2016}\vspace{-1em}\\
\small Thesis topics: resource allocation, communication networks, signal processing \normalsize\\
\cvitem{Professional Engineering Qualification}{Oct 2012}\vspace{-1em}\\
\cvitem{M.Sc. in Telecommunication Engineering}{May 2012}\vspace{-1em}\\	

\vspace{0.5em}
\moveleft\hoffset\vbox{\color{color1}\hrule width\resumewidth height 1pt}\smallskip % Horizontal line after name; adjust line thickness by changing the '1pt'
\vspace{0.5em}

\section{Professional \\and Academic Experience} 

\textbf{Cambridge Touch Technologies (CTT)}, Cambridge, United Kingdom\\
\cvitem{DSP Team Leader}{Jan 2022 - currently}
\cvitem{Principal DSP Engineer}{Jul 2021 - Dec 2021}
\cvitem{Senior DSP Engineer}{Oct 2018 - Jun 2021}
\expbox{CTT is a tech scale-up, developing a cost effective technology to add touch-pressure sensing capabilities to display and non-display surfaces. 

At CTT I currently lead the research, development and implementation of the algorithms that are part of the core of the company IP.

The tech stack I contributed to design and implement comprises Matlab and Python for reference implementations, demos, scripts and glue code, and C for low level, embedded applications.

Main responsibilities include
\begin{itemize}
 \item coordinate algorithm and software development and integration to minimize conflicts between concurrent threads of development
 \item plan and contribute research activities 
 \item provide feedback to help develop the sensor technology
\end{itemize}

\begin{itemize}
 \item I've set up the software versioning and CI infrastructure, based on the Gitlab platform and state of the art dockerised build/test environments on an on-prem cloud.
 \item implemented several real time visualisations and developing environments to demonstrate, familiarise with, and tune DSP algorithms
\end{itemize}
}
\pagebreak
\section{Professional \\and Academic Experience\\(cont'd)} 

\textbf{5G Innovation Centre}, Guildford, United Kingdom\\
\cvitem{Senior Software Engineer}{Nov 2017 - Oct 2018}
\expbox{The 5GIC is a research centre within the University of Surrey, featuring one of the biggest 5G testbeds in the UK with research and standardization activity in close partnership with some of the largest players in the field. The 5GIC's testbed, a complete cellular network, showcases several research ideas, from the physical layer to network concepts.
At the 5GIC I was part of the core network team. My contribution as a software engineer was to develop and maintain the codebase for the core network.}
\vspace {1em}
\textbf{AKYA ltd}, Swindon, United Kingdom\\
\cvitem{DSP Software Engineer}{Dec 2016 - Nov 2017}
\expbox{AKYA's core business was a novel ``dynamically reconfigurable logic'' approach to hardware design, for low power/low cost applications, unlike e.g. FPGA, providing ``just enough'' reconfigurability to meet design requirements. 
At AKYA I contributed as a software engineer to develop a framework that synthesises RT logic from a high-level description of the hardware. My role was to design and develop algorithms and software components that expand and integrate the existing framework, as well as creating tools for testing and data visualization.}

\vspace {1em}
\textbf{DSP Lab, University of Genova}, Genova, Italy\\
\cvitem{Post Doctoral Research Fellow}{Jan 2016 - Nov 2016}
\cvitem{Ph.D. Student}{Jan 2013 - Dec 2015}
\cvitem{Research Fellow}{Oct 2012 - Dec 2012}
\expbox{During my academic experience at the DSP Labs, as Ph.D. student and then research fellow, I carried out academic (research and teaching) activity, along with projects in collaboration with SMEs and some of the key industries of Italy's communications and tech (Telecom Italia, Leonardo). My main area of research were resource allocation and mathematical optimization in communication networks and in signal processing}

\vfill
\moveleft\hoffset\vbox{\color{color1}\hrule width\resumewidth height 1pt}\smallskip % Horizontal line after name; adjust line thickness by changing the '1pt'
\vspace{-0.5em}
\footnotesize
The source to this CV is open and available here: \url{https://github.com/gluzzati/cv}.\\
Feel free to re-use it if you liked it! 

\end{resume}
\end{document}
