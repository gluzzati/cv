\documentclass[mm, 10pt]{res} % Use the res.cls style, the font size can be changed to 11pt or 12pt here
\usepackage[T1]{fontenc}   
\usepackage[left=15mm, right=20mm, top=10mm ,bottom=10mm]{geometry}
%\usepackage{helvet} % Default font is the helvetica postscript font
\usepackage{array}
\usepackage{hyperref}


%\usepackage{newcent} % To change the default font to the new century schoolbook postscript font uncomment this line and comment the one above
\usepackage[rm,sfdefault]{roboto}
\usepackage{xcolor}

\setlength{\textwidth}{5.85in} % Text width of the document
\definecolor{color1}{rgb}{0.4,0.5,0.3}% green

\usepackage{enumitem}
\setitemize{noitemsep,topsep=0pt,parsep=0pt,partopsep=0pt}

\newcommand{\ecvitem}[2]{{\robotocondensed #2}

}

\newcommand{\skilla}[1]{
\begin{tabular}{>{\raggedleft}m{1.6cm} l}
#1&\hspace{-0.5em}{\Large{\color{color1}$\bullet$}{\color{gray!50}$\bullet\bullet$}} \\
\end{tabular}
}
\newcommand{\skillb}[1]{
\begin{tabular}{>{\raggedleft}m{1.6cm} l}
#1&\hspace{-0.5em}{\textbf{\Large{\color{color1}$\bullet\bullet$}{\color{gray!50}$\bullet$}}} \\
\end{tabular}
}
\newcommand{\skillc}[1]{
\begin{tabular}{>{\raggedleft}m{1.6cm} l}
#1&\hspace{-0.5em}{\textbf{\Large{\color{color1}$\bullet\bullet$}{\color{color1}$\bullet$}}} \\
\end{tabular}
}

\newcommand{\expbox}[1]{\vspace{-1.8em}
\noindent\begin{tcolorbox}[boxsep=0pt,top=3pt,left=8pt,right=0pt, bottom=2pt,arc=0pt,auto outer arc,colback=color1!15,colframe=color1!10]
\begin{minipage}[r]{0.95\linewidth} 
\small
#1 
\normalsize
\end{minipage}
\end{tcolorbox}
}

\newcommand{\tinyexpbox}[1]{\vspace{-1.8em}
\noindent\begin{tcolorbox}[boxsep=0pt,top=3pt,left=8pt,right=0pt, bottom=2pt,arc=0pt,auto outer arc,colback=color1!15,colframe=color1!10]
\begin{minipage}[r]{0.95\linewidth} 
\footnotesize
#1 
\normalsize
\end{minipage}
\end{tcolorbox}
}


\newcommand{\cvitem}[2]{{\sl #1} \hfill \textbf{#2}\\}
\newcommand{\detcvitem}[3]{{\sl #1} \hfill \textbf{#2}\\{\hspace*{1em} \small \robotocondensed #3}\\}
\newcommand{\project}[5]{{\sl #1} \hfill \textbf{#2}\\ 
\textbf{Partners:} #3\\
\textbf{Decription:} \small #4\\
\textbf{My role:}\vspace{-1em} #5\par}

\usepackage{fontawesome}
\usepackage[most]{tcolorbox}


\begin{document}

 \hspace{-8.5em} { \Large \robotoslab Giulio Luzzati, Ph.D.}% Your name at the top
 \vspace{-0.5em}

\moveleft\hoffset\vbox{\color{color1}\hrule width\resumewidth height 1pt}\smallskip % Horizontal line after name; adjust line thickness by changing the '1pt'
\vspace{-2em}
%----------------------------------------------------------------------------------------

\begin{resume}
\robotoslab

\section{Personal Information}
Research Software Engineer\\
{\robotocondensed 1b Mill Road, Haddenham, Ely} \\
Phone: {\robotocondensed +44 07397 791131}\\
e-mail: {\robotocondensed giulio.luzzati@live.it}\\
LinkedIn: {\url{https://it.linkedin.com/in/giulio-luzzati-9bb79245}}\\
Born: 19/10/1984, Genova, Italy

\section{Education}
\textbf{University of Genova,} Genova, Italy\\
\cvitem{Ph.D. in Computer Science}{Apr 2016}\vspace{-1em}\\
\small Thesis topics: resource allocation, communication networks, signal processing \normalsize\\
\cvitem{Professional Engineering Qualification}{Oct 2012}\vspace{-1em}\\
\cvitem{M.Sc. in Telecommunication Engineering}{May 2012}\vspace{-1em}\\	

\section{Professional \\and Academic Experience} 
\textbf{Cambridge Touch Technologies (CTT)}, Cambridge, United Kingdom\\
\cvitem{Senior DSP Engineer}{Oct 2018 - Current}
\expbox{CTT is a touch screen technology company, focused on providing cost effective and minimal overhead ways to integrate pressure sensing capabilities to touchscreen technologies. At CTT I design and manage the realization of several projects and tools to diagnose, test and benchmark components of the hardware stack. On top of that, I was tasked with the rearchitecturing and realization of the company's software assets versioning and CI infrastructure.}

\textbf{5G Innovation Centre}, Guildford, United Kingdom\\
\cvitem{Senior Software Engineer}{Nov 2017 - Oct 2018}
\expbox{The 5GIC is a research centre within the University of Surrey, featuring one of the biggest 5G testbeds in the UK and carrying on research and standardization activity in close partnership with the largest players in the field. The 5GIC's testbed is a complete cellular network, used to test and showcase several research ideas ranging from radio technology to network concepts.
At the 5GIC I was part of the core network development team. My contribution as a software engineer was to develop and maintain the code base for the core network.}

\textbf{AKYA ltd}, Swindon, United Kingdom\\
\cvitem{DSP Software Engineer}{Dec 2016 - Nov 2017}
\expbox{AKYA was a small startup whose core business was a novel ``dynamically reconfigurable logic'' approach to hardware design.
AKYA's technology was targeted at low power, low cost applications (as opposed to, e.g., FPGA), providing ``just enough'' reconfigurability to meet the design requirements.\\
At AKYA I contributed as a software engineer to the codebase of a framework that allowed to synthesise logic starting from a very high-level functional description of the hardware. My role in the team was to design and develop algorithms and software components that expand and integrate the existing framework, as well as creating testing and data visualization tools.}

\textbf{DSP Lab, University of Genova}, Genova, Italy\\
\cvitem{Post Doctoral Research Fellow}{Jan 2016 - Nov 2016}
\cvitem{Ph.D. Student}{Jan 2013 - Dec 2015}
\cvitem{Research Fellow}{Oct 2012 - Dec 2012}
\expbox{During my academic experience at the DSP Labs, as Ph.D. student and then research fellow, I carried out academic (research and teaching) activity, along with collaborations with SMEs and some of the key industries of Italy's communications and tech (Telecom Italia, Leonardo). My main area of research were resource allocation and mathematical optimization in communication networks and in signal processing}
% \textbf{National Council of Research (CNR)}, Genova, Italy\\
%{\sl Internship, ISSIA group,  Bachelor Thesis research. } \hfill{\bf  2008}
\pagebreak
\section{In a\\Nutshell}~
\tinyexpbox{3 lines of self introduction\\2 \\ 3}


\section{Technical \\Skills}
\begin{tabular}{cc}
\skillb{C++}&
\skillc{C}\\
\skillc{Python}&
\skillb{Matlab}\\
\skillb{Latex}&
\skilla{Java}\\
\end{tabular}

\footnotesize
\begin{tabular}{lll}
Software Architecture & Build Systems & CI, DevOps \\
Containerization & Embedded Programming & Basic Hardware Diagnostics \\
Service Oriented Architecture & Agile Software Development & GNU/Linux OS \\
\end{tabular}
\normalsize

\footnotesize
\begin{tabular}{l}
Strong experience in computer networks\\
Strong background in signal processing \\
Statistics, mathematical optimization and data science \\
Working knowledge in machine learning \\
Scientific writing and teaching \\
\end{tabular}
\normalsize
\section{Relevant \\Projects}
Signal Injector for Pressure Sensors\\
\cvitem{Project manager, Cambridge Touch Technologies}{2019}
\tinyexpbox{The goal of the project was to realize a signal injector, able to ``simulate'' the behaviour of the physical sensor. An array of DACs at its core, the deliverable consisted in a small box, controlled via REST api, able to physically interface to the pressure sensor amplifier. My role in the project:
\begin{itemize}
\item specified the requirements 
\item designed the high level architecture  
\item implemented driver, API 
\item integration (as slave component) with other existing diagnostic tools
\end{itemize}

The tool proved to be useful and usable, and after the prototype, several units of the tool have been commissioned and are in use by the hardware team.
}


Touchscreen Analyser\\
\cvitem{Project Manager, Cambridge Touch Technologies}{2019}
\tinyexpbox{A tool to perform impedance spectroscopy to diagnose manifacturing defects of pressure sensing enabled displays. 
It automatically generates easy to read reports containing physical measures and inferred features.
My role in the project:
\begin{itemize}
\item designed the flow of operations, data format and structures, and system level architecture 
\item coded most of the software and the GUI 
\item designed and coded signal and data processing algorithms (feature extraction, classification)
\item automated PDF report generation
\end{itemize}
The finished prototype was validated against known samples, and successfully flagged manually undetected defects.}


Human Touch Model \\
\cvitem{Individual Project, Cambridge Touch Technologies}{2020}
\tinyexpbox{This project .. aoroeiroeirpewjfioeofiewjofjeofjfoifjd d

fsdfdsfdsfsdfsd fds fdssdfdsfdsfds fdsf sdfsd fsd f sdf dsf dsfsd }


Quick Http Messages \\
\cvitem{Individual Project, 5GIC}{2018}
\tinyexpbox{This project was a proof of concept to provide a barebone framework to test ideas and quickly prototype in service based architectures (SbA), such as the 5G core network. The core idea was to implement a simple HTTP server over UDP, to minimize latency. Source code for this project available at \url{https://github.com/giuliol/quick-http-messages}.
Additionally, I realized a simple automatic generator able to parse 3GPP OpenAPI compliant YML for the 5G SbA and generate stubs for their REST APIs. 
}



%
%\section{Teaching and Editorial \\ Activity}
%\textbf{University of Genova}, Genova, Italy
%
%\detcvitem{Course of ``Yacht Navigation Support Systems''}{2016}{ Teaching Assistant}
%\detcvitem{Course of ``Telecommunication Networks for Transportation Systems}{2015}{Teaching Assistant}
%\cvitem{Co-Advisor of Master theses on my research topics}{2015 - 2016}
%\vspace{-1em}
%
%\textbf{Higher Technical Institute - Information and\\ Communication Technology  (ITS-ICT)}, Genova, Italy
%
%\detcvitem{Course ``Introduction to programming and signal processing on Android platform''}{2015 - 2016}{Lecturer}
%\detcvitem{Course ``Introduction to Android programming''}{2014}{Lecturer}
%\detcvitem{Course ``Mobile Operating Systems: Linux Kernel''}{2015}{Lecturer}
%\vspace{-1em}
%
%
%\detcvitem{EAI BIGDATACPS 2016 conference}{}{Technical Program Committee member}
%\detcvitem{IEEE ICC 2017 conference}{}{Technical Program Committee member, SAC symposium, Satellite and Space Communication track}
%\detcvitem{IEEE ICC 2016 conference}{}{Technical Program Committee member, SAC symposium, Satellite and Space Communication track}
%\detcvitem{IARIA SPACOMM 2014 conference}{}{Session Chair}\pagebreak

%
%\section{Projects and Industrial Collaborations}
%\project{\small Portable microwave-based device for the differential diagnosis \\of ischemic or hemorrhagic stroke}
%{Jan 2016 - Oct 2016}
%{University of Genova, FOS s.r.l.}
%{ The project deals with the employment of microwave radiation to provide quick and inexpensive diagnostic tools able to support the first responders in case of brain stroke, providing informations on the nature and localization of the stroke. Electromagnetic measures are to be collected via antennas placed on a helmet structure in a multi-static configuration. The expected outcome of the project is a working prototype of the system, which will be realized thanks to the collaboration with FOS s.r.l.
%The project is funded by the Compagnia di San Paolo foundation.
%}
%
%{
%\begin{itemize}[noitemsep,topsep=0pt,parsep=0pt,partopsep=0pt]
%\item Definition of requirements
%\item Research, implementation and experimentation of computational intelligence approaches to the detect the presence of hemorrhages using raw antenna signals (sole contributor)
%\end{itemize}
%} 
%
%
%
%\project{Smarter Networks}
%{Jan 2013 - Dec 2016}
%{AM General Contractor, Mantero Sistemi S.r.l., University of Genova}
%{The aim of the project was to geolocalize and track assets and persons in working areas. Entrances and exits of the warehouses have been monitored with RFID gates by Mantero Sistemi. The tracking of assets and workers within the working areas has been carried out by using Bluetooth Low Energy (BLE) tags. The University of Genoa group has developed two Android applications. The first one, for the foremen that patrol the working areas, periodically performs a Bluetooth scan for revealing the position of assets and workers. The second one is a consultation application that shows the position of the discovered asset on a map.}
%{
%\begin{itemize}[noitemsep,topsep=0pt,parsep=0pt,partopsep=0pt]
%\item Definition of requirements
%\item Smartphone App software development
%\end{itemize}
%} 
%
%\project{JSatSim}
%{2014}
%{University of Genova}
%{The project's outcome is a simple software that allows to simulate the effects of the presence of a satellite link, by shaping the features of the traffic out of a given NIC, in terms of delay, packet loss rate, and bandwidth.
%The user has control over the main parameters of a simple satellite link budget (e.g., the satellite's orbit, the employed modulation and FEC, its elevation, etc.).
%JSatSim has been presented in a demo session during the PSATS international conference.
%The project is open source and it is available at \url{https://github.com/giuliol/jsatsim}
%}
%{
%\begin{itemize}[noitemsep,topsep=0pt,parsep=0pt,partopsep=0pt]
%\item Design and software implementation  (sole contributor)
%\end{itemize}
%}
%
%\project{IANUS}
%{Jan 2013 - Dec 2015}
%{University of Genova}
%{The project focuses on the concept of Ambient Assisted Living and in particular on the possibility to predict possible variations of pathologies afflicting assisted patients, chronically ill and/or with disabilities, via Smartphone terminals. The project has been funded by the program of high technology research and innovation framework of the Liguria Region for the years 2013-2015 (Call PAR FAS 2007/2013 approved with DGR 899 20/07/2012).}
%{
%\begin{itemize}[noitemsep,topsep=0pt,parsep=0pt,partopsep=0pt]
%\item Definition of requirements
%\item Smartphone App software development
%\end{itemize}
%} 
%
%\project{Android Demonstrator}
%{2013}
%{Selex ES, University of Genova}
%{The goal of the collaboration was the realization of a demonstrator of a system capable of transmitting real time video streams able to adapt to  channels characterized by significant packet loss and highly fluctuating bandwidth. The solution has been implemented through two Android application, a transmitter and a receiver. The transmitter performs source coding of the video frames and employs channel coding in order to provide robustness for the stream. 
%The transmitter is able to dynamically adapt the source- and the channel-coder's parameters with the aim to optimize the channel exploitation.
%A simple proof of concept heuristic has been devised to provide the decision rules.}
%{
%\begin{itemize}[noitemsep,topsep=0pt,parsep=0pt,partopsep=0pt]
%\item Definition of requirements
%\item Kernel adaptation to allow RNDIS on Android
%\item Video stream and feedback protocol design (sole contributor)
%\item Software implementation (sole contributor)
%\end{itemize}
%} 
%
%\project{ACONTV}
%{2012 - June 2013}
%{Telecom Italia, University of Genova}
%{The aim of the project was to detect which live television channel is watched by a smartphone user. The task had to be done in an opportunistic way, i.e., without any active collaboration of the user or the broadcaster. An audio fingerprint algorithm has been developed by the University of Genoa for detecting the live TV channel. The algorithm relies on a client-server architecture. The clients have been implemented on Android and iOS devices. The server has been implemented in the Telecom Italia Laboratories located in Turin. An optimization has been carried out for setting fingerprint extraction parameters in order to make the operation lighter in order to less affect the smartphone lifetime.}
%{
%\begin{itemize}[noitemsep,topsep=0pt,parsep=0pt,partopsep=0pt]
%\item Fingerprint algorithm software implementation (C++)
%\item Complete server prototype design and implementation
%\item Optimization of fingerprint extraction algorithm parameters
%\end{itemize}
%} 

%\section{Post-graduate Formation}
%
%\cvitem{``Statistical Signal Processing for Multimedia Forensics and Security''}{May  2015}
%Lecturer: Prof. Fernando P\'erez-Gonz\'alez\\
%Milan (MI), Italy
%
%\cvitem{``First IEEE SPS Italy Chapter Summer School on Signal Processing''}{Sep 2013}
%Riotorto (LI), Italy
%

%\hfill
%\small
\pagebreak
\section{Research \& Publications}

\ecvitem{2016}{Igor Bisio, Fabio Lavagetto, Giulio Luzzati and Andrea Sciarrone, ``A Novel Active Warden Technique for Image Steganography'', accepted IEEE GLOBECOM 2016. }
\ecvitem{2016}{I. Bisio, A. Fedeli, F. Lavagetto, G. Luzzati, M. Pastorino, A. Randazzo, and E. Tavanti, ``Brain Stroke Detection by Microwave Imaging Systems: Preliminary Two-Dimensional Numerical Simulations'', submitted to 2016 IEEE International Conference on Imaging Systems and Techniques (IST 2016)}
\ecvitem{2016}{Igor Bisio, Alessandro Fedeli, Fabio Lavagetto, Giulio Luzzati, 
Matteo Pastorino,  Andrea Randazzo, and Emanuele Tavanti, ``Hemorrhagic Brain Stroke Detection by using Microwaves: Preliminary Two-dimensional Reconstructions'', IV Convegno Nazionale ``Interazione tra Campi Elettromagnetici e Biosistemi'', Milano, 4-6 July 2016. }

\ecvitem{2016}{ Igor Bisio, Fabio Lavagetto, Giulio Luzzati, ``Cooperative Application Layer Joint Video Coding in the Internet of Remote Things'', submitted to the IEEE Internet of Things Journal.}

\ecvitem{2015}{Igor Bisio, Giulio Luzzati and Andrea Sciarrone, ``Cell-ID Meter App: a Tester for Coverage Maps Localization Proofs in Forensic'' Investigations, 7th IEEE International Workshop on Information Forensics and SecurityRome, Italy, 16-19 November, 2015}
\ecvitem{2015}{Igor Bisio and Stefano Delucchi and Fabio Lavagetto and Giulio Luzzati and
Mario Marchese, ``Cooperative Application Layer Joint Coding and Rate Allocation Techniques for Video Transmissions over Satellite Channels through Smartphones'', accepted to IEEE ICC 2015 SAC - Satellite and Space Communications (ICC'15 (01) SAC6-SSC)}
\ecvitem{2014}{Igor Bisio, Fabio Lavagetto, Giulio Luzzati, Mario Marchese, ``Smartphones Apps Implementing a Heuristic Joint Coding for Video Transmissions over Mobile Networks'', International Journal of Mobile Networks and Applications (MONET).}
\ecvitem{2014}{Igor Bisio, Aldo Grattarola, Fabio Lavagetto, Giulio Luzzati, Mario Marchese, ``Application Layer Source-Channel Video Coding for Transmission with Smartphones over Satellite Channel'', Proc. The Sixth International Conference on Advances in Satellite and Space Communications (SPACOMM), February 23 - 27, 2014 - Nice, France.}
\ecvitem{2014}{Igor Bisio, Fabio Lavagetto, Giulio Luzzati, Mario Marchese,  ``Smartphones Apps Implementing a Heuristic Joint Coding for Video Transmissions over Mobile Networks'', 6th International Conference on Personal Satellite Services, July 2014, Genoa, Italy}
\ecvitem{2014}{Igor Bisio, Aldo Grattarola, Fabio Lavagetto, Giulio Luzzati, Mario Marchese, ``Performance Evaluation of Application Layer Joint Coding for Video Transmission with Smartphones Over Terrestrial/Satellite Emergency Networks'', Proc. IEEE International Conference on Communications 2014, ICC 2014, 10 – 14 June 2014, Sydney, Australia - Best Paper Award winner}
\ecvitem{2012}{Bisio, I.; Delfino, A.; Luzzati, G.; Lavagetto, F.; Marchese, M.; Fra, C.; Valla, M., ``Opportunistic estimation of television audience through smartphones,'' Performance Evaluation of Computer and Telecommunication Systems (SPECTS), 2012 International Symposium on , vol., no., pp.1,5, 8-11 July 2012}


\end{resume}
\end{document}
