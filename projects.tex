\section{Projects and Industrial Collaborations}
\project{\small Portable microwave-based device for the differential diagnosis \\of ischemic or hemorrhagic stroke}
{Jan 2016 - Oct 2016}
{University of Genova, FOS s.r.l.}
{ The project deals with the employment of microwave radiation to provide quick and inexpensive diagnostic tools able to support the first responders in case of brain stroke, providing informations on the nature and localization of the stroke. Electromagnetic measures are to be collected via antennas placed on a helmet structure in a multi-static configuration. The expected outcome of the project is a working prototype of the system, which will be realized thanks to the collaboration with FOS s.r.l.
The project is funded by the Compagnia di San Paolo foundation.
}

{
\begin{itemize}[noitemsep,topsep=0pt,parsep=0pt,partopsep=0pt]
\item Definition of requirements
\item Research, implementation and experimentation of computational intelligence approaches to the detect the presence of hemorrhages using raw antenna signals (sole contributor)
\end{itemize}
} 



\project{Smarter Networks}
{Jan 2013 - Dec 2016}
{AM General Contractor, Mantero Sistemi S.r.l., University of Genova}
{The aim of the project was to geolocalize and track assets and persons in working areas. Entrances and exits of the warehouses have been monitored with RFID gates by Mantero Sistemi. The tracking of assets and workers within the working areas has been carried out by using Bluetooth Low Energy (BLE) tags. The University of Genoa group has developed two Android applications. The first one, for the foremen that patrol the working areas, periodically performs a Bluetooth scan for revealing the position of assets and workers. The second one is a consultation application that shows the position of the discovered asset on a map.}
{
\begin{itemize}[noitemsep,topsep=0pt,parsep=0pt,partopsep=0pt]
\item Definition of requirements
\item Smartphone App software development
\end{itemize}
} 

\project{JSatSim}
{2014}
{University of Genova}
{The project's outcome is a simple software that allows to simulate the effects of the presence of a satellite link, by shaping the features of the traffic out of a given NIC, in terms of delay, packet loss rate, and bandwidth.
The user has control over the main parameters of a simple satellite link budget (e.g., the satellite's orbit, the employed modulation and FEC, its elevation, etc.).
JSatSim has been presented in a demo session during the PSATS international conference.
The project is open source and it is available at \url{https://github.com/giuliol/jsatsim}
}
{
\begin{itemize}[noitemsep,topsep=0pt,parsep=0pt,partopsep=0pt]
\item Design and software implementation  (sole contributor)
\end{itemize}
}

\project{IANUS}
{Jan 2013 - Dec 2015}
{University of Genova}
{The project focuses on the concept of Ambient Assisted Living and in particular on the possibility to predict possible variations of pathologies afflicting assisted patients, chronically ill and/or with disabilities, via Smartphone terminals. The project has been funded by the program of high technology research and innovation framework of the Liguria Region for the years 2013-2015 (Call PAR FAS 2007/2013 approved with DGR 899 20/07/2012).}
{
\begin{itemize}[noitemsep,topsep=0pt,parsep=0pt,partopsep=0pt]
\item Definition of requirements
\item Smartphone App software development
\end{itemize}
} 

\project{Android Demonstrator}
{2013}
{Selex ES, University of Genova}
{The goal of the collaboration was the realization of a demonstrator of a system capable of transmitting real time video streams able to adapt to  channels characterized by significant packet loss and highly fluctuating bandwidth. The solution has been implemented through two Android application, a transmitter and a receiver. The transmitter performs source coding of the video frames and employs channel coding in order to provide robustness for the stream. 
The transmitter is able to dynamically adapt the source- and the channel-coder's parameters with the aim to optimize the channel exploitation.
A simple proof of concept heuristic has been devised to provide the decision rules.}
{
\begin{itemize}[noitemsep,topsep=0pt,parsep=0pt,partopsep=0pt]
\item Definition of requirements
\item Kernel adaptation to allow RNDIS on Android
\item Video stream and feedback protocol design (sole contributor)
\item Software implementation (sole contributor)
\end{itemize}
} 

\project{ACONTV}
{2012 - June 2013}
{Telecom Italia, University of Genova}
{The aim of the project was to detect which live television channel is watched by a smartphone user. The task had to be done in an opportunistic way, i.e., without any active collaboration of the user or the broadcaster. An audio fingerprint algorithm has been developed by the University of Genoa for detecting the live TV channel. The algorithm relies on a client-server architecture. The clients have been implemented on Android and iOS devices. The server has been implemented in the Telecom Italia Laboratories located in Turin. An optimization has been carried out for setting fingerprint extraction parameters in order to make the operation lighter in order to less affect the smartphone lifetime.}
{
\begin{itemize}[noitemsep,topsep=0pt,parsep=0pt,partopsep=0pt]
\item Fingerprint algorithm software implementation (C++)
\item Complete server prototype design and implementation
\item Optimization of fingerprint extraction algorithm parameters
\end{itemize}
} 
